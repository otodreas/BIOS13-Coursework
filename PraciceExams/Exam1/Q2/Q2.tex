% Options for packages loaded elsewhere
% Options for packages loaded elsewhere
\PassOptionsToPackage{unicode}{hyperref}
\PassOptionsToPackage{hyphens}{url}
\PassOptionsToPackage{dvipsnames,svgnames,x11names}{xcolor}
%
\documentclass[
  a4paper,
]{report}
\usepackage{xcolor}
\usepackage{amsmath,amssymb}
\setcounter{secnumdepth}{-\maxdimen} % remove section numbering
\usepackage{iftex}
\ifPDFTeX
  \usepackage[T1]{fontenc}
  \usepackage[utf8]{inputenc}
  \usepackage{textcomp} % provide euro and other symbols
\else % if luatex or xetex
  \usepackage{unicode-math} % this also loads fontspec
  \defaultfontfeatures{Scale=MatchLowercase}
  \defaultfontfeatures[\rmfamily]{Ligatures=TeX,Scale=1}
\fi
\usepackage{lmodern}
\ifPDFTeX\else
  % xetex/luatex font selection
\fi
% Use upquote if available, for straight quotes in verbatim environments
\IfFileExists{upquote.sty}{\usepackage{upquote}}{}
\IfFileExists{microtype.sty}{% use microtype if available
  \usepackage[]{microtype}
  \UseMicrotypeSet[protrusion]{basicmath} % disable protrusion for tt fonts
}{}
\makeatletter
\@ifundefined{KOMAClassName}{% if non-KOMA class
  \IfFileExists{parskip.sty}{%
    \usepackage{parskip}
  }{% else
    \setlength{\parindent}{0pt}
    \setlength{\parskip}{6pt plus 2pt minus 1pt}}
}{% if KOMA class
  \KOMAoptions{parskip=half}}
\makeatother
% Make \paragraph and \subparagraph free-standing
\makeatletter
\ifx\paragraph\undefined\else
  \let\oldparagraph\paragraph
  \renewcommand{\paragraph}{
    \@ifstar
      \xxxParagraphStar
      \xxxParagraphNoStar
  }
  \newcommand{\xxxParagraphStar}[1]{\oldparagraph*{#1}\mbox{}}
  \newcommand{\xxxParagraphNoStar}[1]{\oldparagraph{#1}\mbox{}}
\fi
\ifx\subparagraph\undefined\else
  \let\oldsubparagraph\subparagraph
  \renewcommand{\subparagraph}{
    \@ifstar
      \xxxSubParagraphStar
      \xxxSubParagraphNoStar
  }
  \newcommand{\xxxSubParagraphStar}[1]{\oldsubparagraph*{#1}\mbox{}}
  \newcommand{\xxxSubParagraphNoStar}[1]{\oldsubparagraph{#1}\mbox{}}
\fi
\makeatother


\usepackage{longtable,booktabs,array}
\usepackage{calc} % for calculating minipage widths
% Correct order of tables after \paragraph or \subparagraph
\usepackage{etoolbox}
\makeatletter
\patchcmd\longtable{\par}{\if@noskipsec\mbox{}\fi\par}{}{}
\makeatother
% Allow footnotes in longtable head/foot
\IfFileExists{footnotehyper.sty}{\usepackage{footnotehyper}}{\usepackage{footnote}}
\makesavenoteenv{longtable}
\usepackage{graphicx}
\makeatletter
\newsavebox\pandoc@box
\newcommand*\pandocbounded[1]{% scales image to fit in text height/width
  \sbox\pandoc@box{#1}%
  \Gscale@div\@tempa{\textheight}{\dimexpr\ht\pandoc@box+\dp\pandoc@box\relax}%
  \Gscale@div\@tempb{\linewidth}{\wd\pandoc@box}%
  \ifdim\@tempb\p@<\@tempa\p@\let\@tempa\@tempb\fi% select the smaller of both
  \ifdim\@tempa\p@<\p@\scalebox{\@tempa}{\usebox\pandoc@box}%
  \else\usebox{\pandoc@box}%
  \fi%
}
% Set default figure placement to htbp
\def\fps@figure{htbp}
\makeatother





\setlength{\emergencystretch}{3em} % prevent overfull lines

\providecommand{\tightlist}{%
  \setlength{\itemsep}{0pt}\setlength{\parskip}{0pt}}



 


\makeatletter
\@ifpackageloaded{caption}{}{\usepackage{caption}}
\AtBeginDocument{%
\ifdefined\contentsname
  \renewcommand*\contentsname{Table of contents}
\else
  \newcommand\contentsname{Table of contents}
\fi
\ifdefined\listfigurename
  \renewcommand*\listfigurename{List of Figures}
\else
  \newcommand\listfigurename{List of Figures}
\fi
\ifdefined\listtablename
  \renewcommand*\listtablename{List of Tables}
\else
  \newcommand\listtablename{List of Tables}
\fi
\ifdefined\figurename
  \renewcommand*\figurename{Figure}
\else
  \newcommand\figurename{Figure}
\fi
\ifdefined\tablename
  \renewcommand*\tablename{Table}
\else
  \newcommand\tablename{Table}
\fi
}
\@ifpackageloaded{float}{}{\usepackage{float}}
\floatstyle{ruled}
\@ifundefined{c@chapter}{\newfloat{codelisting}{h}{lop}}{\newfloat{codelisting}{h}{lop}[chapter]}
\floatname{codelisting}{Listing}
\newcommand*\listoflistings{\listof{codelisting}{List of Listings}}
\makeatother
\makeatletter
\makeatother
\makeatletter
\@ifpackageloaded{caption}{}{\usepackage{caption}}
\@ifpackageloaded{subcaption}{}{\usepackage{subcaption}}
\makeatother
\usepackage{bookmark}
\IfFileExists{xurl.sty}{\usepackage{xurl}}{} % add URL line breaks if available
\urlstyle{same}
\hypersetup{
  pdftitle={Q2},
  colorlinks=true,
  linkcolor={blue},
  filecolor={Maroon},
  citecolor={Blue},
  urlcolor={Blue},
  pdfcreator={LaTeX via pandoc}}


\title{Q2}
\author{}
\date{}
\begin{document}
\maketitle


\chapter{Dynamics of a spatially structured
population}\label{dynamics-of-a-spatially-structured-population}

\section{Part A: Non-trivial equilibrium population
size}\label{part-a-non-trivial-equilibrium-population-size}

The equilibrium population sizes are defined as the values of \(n_1\)
for which the change in population size over time
\(\frac{dn_1}{dt} = 0\). For this condition to be true, one of the two
terms \(r_1 - k n_1^2\) or \(n_1\) must equal \(0\), since their product
is the change in population size over time \(\frac{dn_1}{dt}\).

Thus, the non-trivial solution is

\[
r_1 - k n_1^2 = 0 \Rightarrow n_1 = \sqrt{\frac{r_1}{k}}
\]

\section{Part B: Prove stability of
equilibrium}\label{part-b-prove-stability-of-equilibrium}

For the equilibrium to be stable, the first derivative of the model (the
second derivative if the model represents population rather than change
in population) must be negative. If this condition is met, population
sizes immediately below the equilibrium will have a positive
\(\frac{dn_1}{dt}\), meaning that the population will grow towards the
equilibrium size, while population sizes immediately above the
equilibrium will have a negative \(\frac{dn_1}{dt}\) and shrink towards
the equilibrium size.

To determine the stability of the equilibrium, we first rearrange
\(\frac{dn_1}{dt}\) at the non-trivial equilibrium, where (I've
``renamed'' it to \(f'(n_1)\) for clarity).

\[
\begin{align}
    f'(n_1) &= \left(r_1 - k \sqrt{\frac{r_1}{k}} \right) \sqrt{\frac{r_1}{k}} \\
    &= r_1 \sqrt{\frac{r_1}{k}} - k \frac{r_1}{k} \\
    &= r_1 \left(r_1^{\frac{1}{2}} k^{\frac{-1}{2}} \right) - k \left(r_1 k^{\frac{-1}{2}} \right) \\
    &= r_1^{\frac{3}{2}} k^{\frac{-1}{2}} - k^{\frac{1}{2}} r_1
\end{align}
\]

We can now get the second derivative \(f''(n_1)\)

\[
\begin{align}
    f''(n_1) &= \frac{3}{2} r_1^{\frac{1}{2}} \times \frac{-1}{2} k^{\frac{-3}{2}} - \frac{1}{2} k^{\frac{-1}{2}} \times 1 \\
    &= - \frac{3 \sqrt{r_1}}{4 \sqrt{k^3}} - \frac{1}{2 \sqrt{k}}
\end{align}
\]

Since both \(r_1\) and \(k\) are positive, both terms of \(f''(n_1)\)
are negative, and since the exponentiation performed on \(r_1\) and
\(k\) cannot produce a negative result, there are no possible values for
\(r_1\) and \(k\) that could make \(f''(n_1)\) positive. Therefore

\[
f''(n_1) < 0
\]

When

\[
n_1 = \sqrt{\frac{r_1}{k}}
\]

Thus, the non-trivial equilibrium is stable.

\section{\texorpdfstring{Part C: Plot
\(\frac{dn}{dt}\)}{Part C: Plot \textbackslash frac\{dn\}\{dt\}}}\label{part-c-plot-fracdndt}

The answer is provided in \texttt{Scripts/2C.R}.

\section{Part D: Write new equations}\label{part-d-write-new-equations}

Since \(m\) individuals leave population 1 for population 2 per unit
time, the system of equations that describes the population dynamics of
\(n_1\) and \(n_2\) is defined as

\[
\begin{cases}
\begin{align}
    \frac{dn_1}{dt} &= \left( r_1 - k n_1^2 \right) n_1 - m n_1 \\
    \frac{dn_2}{dt} &= \left( r_2 - k n_2^2 \right) n_2 + m n_1
\end{align}
\end{cases}
\]

\section{Part E: New equilibrium of population
1}\label{part-e-new-equilibrium-of-population-1}

Since the equilibrium is defined as the value of \(n_1\) when
\(\frac{dn_1}{dt} = 0\), we assert the equation below and solve for
\(n_1\).

\[
\begin{align}
    \left( r_1 - k n_1^2 \right) n_1 - m n_1 &= 0 \\
    \left( r_1 - k n_1^2 \right) n_1 &= m n_1 \\
    r_1 - k n_1^2 &= m \\
    n_1 &= \sqrt{\frac{m - r_1}{-k}}
\end{align}
\]

Thus, when the size of population 1 is \(\sqrt{\frac{m - r_1}{-k}}\),
the population is in equilibrium.

\section{\texorpdfstring{Part F: Value of \(m\) where population 1 goes
extinct}{Part F: Value of m where population 1 goes extinct}}\label{part-f-value-of-m-where-population-1-goes-extinct}

For a given population size \(n_1\), the parameter \(m\) must be large
enough to make \(\frac{dn_1}{dt}\) negative. Thus, the term \(m n_1\)
must be greater than the term \(\left(r_1 - k n_1^2 \right) n_1\). We
can solve by asserting the equation below and solving for \(m\).

\[
\begin{align}
    m n_1 &> \left( r_1 - k n_1^2 \right) n_1 \\
    m &> r_1 - k n_1^2
\end{align}
\]

Thus, if \(m > r_1 - k n_1^2\), the population goes extinct.

\section{Part G: Simulate coupled
dynamics}\label{part-g-simulate-coupled-dynamics}

The answer is provided in \texttt{Scripts/2G.R}.




\end{document}
